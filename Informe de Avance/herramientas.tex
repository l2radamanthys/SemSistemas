\chapter{Herramientas a Utilizar}

Una de las herramientas que se utilizar\'a es la Herramienta Case WebRatio que 
es una  herramienta  orientada  justamente  a  WebML (la  metodolog\'{\i}a  que  
estudiar\'e  y aplicar\'e durante el desarrollo del presente seminario) la 
misma se la puede descargar desde el sitio 
\url{http://www.webratio.com/portal/content/es/descargas}. 

Esta herramienta permite desarrollar todos los modelos que la metodolog\'{\i}a 
exige, adem\'as brinda la opci\'on de generar la aplicaci\'on que es algo muy 
interesante ya que se puede observar si en realidad los modelos reflejan lo 
que se pretende tanto en la parte de interfaz como en la funcional. 


Esta  herramienta  CASE para poder funcionar requiere:  

\begin{itemize}
    \item Sistema Operativo: Windows 7, Windows XP, Mac OS X, Linux
    \item Memoria RAM de 512 Mb o superior
    \item Espacio Minimo en el Disco 1 Gb o superior
    \item Resolucion de Pantalla 1024x768 px 
    \item Java Development Kit JDK JRockit 1.6.0\_20 o superior
    \item Apache Tomcat
    \item (Opcional) Plugin Apache Tomcat para WebRatio basado en Eclipse 
\end{itemize}


Para mayor informacion acerca de la instalacion de WebRatio puede consultar 
cite{Instalacion WebRatio}.


Es muy conveniente la utilizaci\'on de esta herramienta por el simple hecho de 
que fue  desarrollada  por  el  grupo  precursor  de  esta  metodolog\'{\i}a,  
adem\'as  de  ser  una herramienta bastante intuitiva y de f\'acil 
utilizaci\'on. 


Tambi\'en se requiere utilizar el servidor Web  Apache Tomcat ya que WebRatio 
genera paginas JSP y son interpretadas por este servidor. Se puede instalar
junto con WebRatio. 
